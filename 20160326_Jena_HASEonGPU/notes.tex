%
% Template for videoprojector foils according to the HZDR corporate design
% 
% (C) 2011 Alexander Grahn, HZDR, Institut fuer Sicherheitsforschung
%
% Version 20131126
%
\documentclass[
%  aspectratio=43, %default screen size (4:3); other values: 1610, 169, 149, 54, 32
  smaller, %uncomment this to fit more text onto the slides (similar to the CD suggestions)
  intlimits %put limits of integrals above and below the integral symbol
]{beamer}

\usepackage[UKenglish]{babel}

%my favourite packages
\usepackage[latin1]{inputenc} %type in accented characters easily, i. e. � instead of \"a 
\usepackage{ragged2e}\RaggedRight
\usepackage{fancyvrb}
\usepackage{ifthen}
\setbeamercolor{alerted text}{fg=hzdr-orange}  %\alert{text to be highlighted},
                                               %\begin{alertenv} ... \end{alertenv}
                                               %\begin{alertblock}{block title (highlighted)}
                                               %  ...
                                               %\end{alertblock}

\setbeamercolor{example text}{fg=hzdr-darkblue}%\begin{exampleblock}{block title (highlighted)}
                                               %  ...
                                               %\end{exampleblock}
%%%%%%%%%%%%%%%%%%%%%%%%%%%%%%%%%%%%%%%%%%%%%%%%%%%%%%%%%%%%%%%%%%%%%%%%%%%%%%%%%%%%%%%%%

%hzdr theme
\makeatletter
\def\input@path{{beamerthemehzdr/}}
\makeatother
\usetheme{hzdr}

%%%%%%%%%%%% title, author, date, institute, additional logos %%%%%%%%%%%%%%%%%%%%%%%%%%%
\title{HASEonGPU - NOTES}
\subtitle{An Open-Source ASE code for calculating the gain in high power laser media on GPU clusters}% may be left empty, of course

%[short version of authors to be used in footlines]{long version in the title page}
\author[C. Eckert, E. Zenker et al.]{
\texorpdfstring{\underline{Carlchristian Eckert}\inst{1}}{},
\texorpdfstring{\underline{Erik Zenker}\inst{1}}{}
}


%\date{\today}
\date{15th July 2014}

\institute[
 %short version of main authors' institute, used in the footline
 \iflanguage{ngerman}{%
   Institut f�r Strahlenphysik%
 }{%
   Institute of Radiation Physics%
 }%
]{%
 %used in the title page
 \iflanguage{ngerman}{%
   \inst{1}Helmholtz-Zentrum Dresden-Rossendorf, Institut f�r Strahlenphysik\\
 }{%
   \inst{1}Helmholtz-Zentrum Dresden-Rossendorf, Institute of Radiation Physics\\
 }%
}
\hypersetup{%
  breaklinks,colorlinks,
  %pdfpagemode=FullScreen,
  pdfhighlight=/P,
  linkcolor=hzdr-darkblue,
  linkcolor=hzdr-darkblue,
  anchorcolor=hzdr-darkblue,
  citecolor=hzdr-darkblue,
  filecolor=hzdr-darkblue,
  menucolor=hzdr-darkblue,
  runcolor=hzdr-darkblue,
  urlcolor=hzdr-darkblue
}


\begin{document}

\frame{\titlepage}
\begin{frame}{Physical Process}
  \begin{itemize}
    \item gain medium is a central compontent of a laser
    \item energy is stored in the medium through "pumping": energy is used to bring electrons in an excited state
    \item if these excited electrons are hit by a photon, they drop to a lower energy level
    \item a photon is emitted, which follows the same direction and phase as the stimulating photon
    \item $\rightarrow$ the stimulating light is amplified
    \item excited electrons have a half-life time: they will spontaneously fall to a lower energy level
    \item this drop also produces a photon with a random direction
    \item this spontaneously emitted photon also gets amplified
    \item the amplification of these random photons is called Amplified Spontaneous Emission
  \end{itemize}
\end{frame}

\begin{frame}{ASE experiment}
  \begin{itemize}
    \item How does this process influence a laser?
    \item graphic: the amplification of a stimulating laser after different pumping durations
    \item Ideal case without ASE: more pumping $\rightarrow$ more excited electrons $\rightarrow$ more amplification
    \item When the pumping process stops, the spontaneous emission will reduce the energy in the medium again $\rightarrow$ less amplification
    \item Experimental measurement shows much less amplification
    \item ASE will cause more excited electrons to emit a photon
    \item This loss in energy reduces the possible amplification (light blue is the difference) 
  \end{itemize}
\end{frame}

\begin{frame}{ASE video}
  \begin{itemize}
    \item The ASE can be simulated and visualized for the whole gain medium
    \item beginning: ASE increases with the amount of energy in the medium
    \item after pump stops, ASE decreases as less energy is contained in the medium
  \end{itemize}
\end{frame}

\begin{frame}{ASE-temperature equilibrium}
  \begin{itemize}
    \item to design new laser systems and (e.g. to improve energy efficiency),
      it is important to know the precise values for ASE
    \item The process to calculate such an ASE distribution is usually very time-consuming (takes days)
    \item For a precise result, the temperature influence must be considered
    \item ASE is a process of energy loss $\rightarrow$ increases temperature in medium (rather fast)
    \item Temperature changes influence the optical and mechanical properties of the gain medium itself (rather fast)
    \item These changes in the medium need to be considered when calculating ASE
    \item it's an iterative process
    \item slowed down by ASE calculation
  \end{itemize}
\end{frame}

\begin{frame}{ASE as MonteCarlo}
  \begin{itemize}
    \item ASE can be examined for a single sample point in the gain medium
    \item For this sample point, ASE can be modeled as a multidimensional integral
    \item This is solvable through MC:
    \item graphic: Gain medium, meshed into prisms. each prism can have different properties
    \item Similar to spontaneous emission, random positions in the medium are
      selected and a photon is strated towards the sample point
    \item While travelling through the medium, the photons are amplified
    \item The path and the amplification can be simulated with ray tracing techniques
    \item When a ray passes through a prism, it is amplified according to the material parameters at this location
    \item Total ASE at the position of a sample point: average of all ASE influences 
  \end{itemize}
\end{frame}

\begin{frame}{Monte Carlo is parallel}
  \begin{itemize}
    \item This Monte Carlo approach is very well suited for parallelization:
      each ray is completely independent, each sample point as well
    \item We used CUDA
    \item Perfect scenario for current accelerators 
    \item Each CUDA thread computes path+amplification of a single ray
    \item Each thread needs lots of data for the traversed prisms $\rightarrow$ many accesses to global memory
    \item Solution: rays and threads are sorted in a way, that allows
      successive threads (block/warp) to start their rays from similar
      locations $\rightarrow$ caching
    \item XeonPhi claims to have very good caching, might be interesting
  \end{itemize}
\end{frame}

\begin{frame}{MSE}
  \begin{itemize}
    \item Problem: how many spontaneous emissions are needed to simulate the physical process?
    \item No clear answer: depends on the sample point
      the ray. Certain areas might need more rays than others
    \item General idea: low variance is a good indication for a stable/correct result
    \item Solution: MSE is used to rate the quality of the simulation
    \item Answer: MC is precise enough, when each sample point is simulated with a low MSE
  \end{itemize}
\end{frame}

\begin{frame}{IS}
  \begin{itemize}
    \item Now, some techniques to reduce the MSE for a given sample point (red)
    \item Naive approach: uniform distribution of photons.
    \item left: starting positions of rays for simulating the given sample point
    \item right: Histogram of resulting MSE values, when whole gain medium is
      simulated with the technique
    \item observation: many sample points produce a quite high MSE
    \item in this example: MSE of 0.005 is still acceptable. more is considered bad
    \item Observation: rays from certain areas have a much higher influence on the result
    \item Idea: increase photon density in those areas and weigh them appropriately
    \item result: MSE could be reduced for some sample points, more precise
      simulation with the same sampling resolution
  \end{itemize}
\end{frame}

\begin{frame}{AS}
  \begin{itemize}
    \item Some sample points will still produce high MSE values, even when using IS
    \item if such a sample point is discovered, more photons/rays are
      generated, to improve resolution. (red)
    \item Number of rays can be increased, until the MSE of the sample point
      crosses a threshold
    \item for the whole gain medium, this means that no sample points are left
      with a high MSE
    \item Simulation is now precise enough!
    \item however, sometimes a LOT of rays is necessary
  \end{itemize}
\end{frame}

\begin{frame}{RS}
  \begin{itemize}
    \item This huge number of rays on a few sample points is very expensive
    \item Observation: sometimes, a bad random seed can cause very high MSE
      values, even with many rays
    \item Idea: instead of more rays, restart with different rays.
    \item for most sample points, this works perfectly to get low MSE with less rays
    \item In the example, the histogram shows that most sample points do NOT need to be sampled with $10^8$ rays.
    \item Actually a difference of $3\cdot10^{10}$ rays, reducing simulation time by 50\%, same precision
  \end{itemize}
\end{frame}

\begin{frame}{MPI}
  \begin{itemize}
    \item All things considered so far were about simulating a sample point on a GPU
    \item This process is to be repeated for each sample point
    \item The interesting thing: all sample points are independent.
    \item This is suited to be distributed on a cluster system
    \item each GPU holds all the necessary data about the mesh (as in the previous cases)
    \item each GPU COULD simulate the whole gain medium
    \item additional Head Node which uses MPI to tell the GPU which sample point to
      simulate next (as soon as the GPU is idle)
    \item it is possible to add more GPUs/nodes, which will receive sample points to simulate
  \end{itemize}
\end{frame}

\begin{frame}{Scaling on Clusters}
  \begin{itemize}
    \item Since the Head Node assigns new work as soon as a GPU is idle, this
      provides load balancing between different sample points (some take longer
      than others) 
    \item Measurements with up to 64 GPUs show almost perfect strong scaling
    \item Simply add more GPUs to improve performance
  \end{itemize}
\end{frame}

\begin{frame}{Validation}
  \begin{itemize}
    \item How good is the simulation and the approach with MSE as a quality control
    \item As a reminder: amplification plot from the real experiment
    \item Now: high precision simulation, is able to predict the outcome of
      experiments very closely
    \item Takes about 7h on 1 GPU
  \end{itemize}
\end{frame}

\begin{frame}{Speedup VS CPU}
  \begin{itemize}
    \item As a comparison: same algorithm as it was initially implemented as a
      single-threaded program on a CPU
    \item would take about 4 weeks
    \item For larger gain media, a lot longer $\rightarrow$ not really practical
    \item Speedup about 2 orders of magnitude
  \end{itemize}
\end{frame}

\begin{frame}{Conclusion}
  \begin{itemize}
    \item As seen, the HASEonGPU software is able to predict ASE very
      precisely, since it actually simulated the physical process, including
      effects like reflections and polychromatic portions of the laser light
    \item The software is a lot faster than current CPU implementations and
      will scale very well.
    \item This speedup allows to determine ASE for current gain media very fast
    \item Looking into the future, this allows to simulate upcomping high-power
      laser systems (for example petawatt lasers) in a reasonable amount of time
  \end{itemize}
\end{frame}

\begin{frame}{Github}
  \begin{itemize}
    \item Thank you for your attention
    \item And in case you are interested in this software, it will soon be
      released on GitHub as Open Source 
  \end{itemize}
\end{frame}



\end{document}
